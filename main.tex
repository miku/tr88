\documentclass{article}
\usepackage{graphicx}

\begin{document}

\title{A NEW HASHING METHOD WITH APPLICATION FOR GAME PLAYING}
\author{Albert L. Zobrist}
\maketitle

\begin{abstract}
A general method of hash coding is described together with an application for
programs which play board games such as checkers, chess, and GO. An auxiliary
method which detects retrieval errors is proposed. The error rate can be
precisely controlled depending upon how much space in the hash table is
devotedto the auxiliary method.
\end{abstract}

\section*{Introduction}

When a computer program stores an item in a large table, subsequent reference
or retrieval of that item may necessitate a search of the table. This is so
unless a table address for the item can be calculated in a systematic fashion
from the item itself. A function which converts items into addresses is called a \underline{hashing algorithm}, and the resulting table a \underline{hash table}. The application of these functions is the subject of a small and specialized literature, but will soon be covered by Knuth\cite{knuth}.

\begin{thebibliography}{9}
\bibitem{knuth}
Knuth, D. \underline{The Art of Computer Programming}, vol. 3 (to appear)
\end{thebibliography}


\end{document}

