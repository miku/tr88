\documentclass{article}
\usepackage{graphicx}
\usepackage{hyperref}
\usepackage[fleqn]{mathtools}

\date{April 1970}

\begin{document}

\title{A NEW HASHING METHOD WITH APPLICATION FOR GAME PLAYING\\ {\small Technical Report \#88}}
\author{Albert L. Zobrist}
\maketitle

\begin{abstract}
A general method of hash coding is described together with an application for
programs which play board games such as checkers, chess, and GO. An auxiliary
method which detects retrieval errors is proposed. The error rate can be
precisely controlled depending upon how much space in the hash table is
devoted to the auxiliary method.
\end{abstract}

\section*{Introduction}

When a computer program stores an item in a large table, subsequent reference
or retrieval of that item may necessitate a search of the table. This is so
unless a table address for the item can be calculated in a systematic fashion
from the item itself. A function which converts items into addresses is called
a \underline{hashing algorithm}, and the resulting table a \underline{hash
table}. The application of these functions is the subject of a small and
specialized literature, but will soon be covered by Knuth\cite{knuth}.

A good hashing function has two properties: randomness in the distribution of
addresses and speed of operation. Two items may produce the same hash address.
We shall call such an occurrence a \underline{clash}. A \underline{clash}
cannot be detected unless the items or derived quantities are stored in the
table. The stored quantities which may be used to resolve clashes will be called
\underline{keys}.

If a game playing program uses lookahead and evaluation followed by minimaxing
to find its move, and if the alternate move sequences give frequent duplication
of board configurations, then it might be useful to store the backed up values
of the board configurations encountered. A hash table would enable fast
retrieval of these values. The Greenblatt chess program\cite{greenblatt}, one
of the most successful game players, uses a 32K hash table to store such
information. A mild acquaintance with game playing programs is assumed for what
follows.

Suppose that a board configuration is reached at depth 5 which has already been
fully evaluated at depth 3. Then the value from the node at depth 3 may be
given to the node at depth 5 with no further lookahead or evaluation
necessary. Since most lookahead algorithms require search to a fixed or
calculated depth under the assumption that backed up values are more reliable,
the values retrieved in this manner should come from the same depth or closer
to the top of the tree. If $\alpha$--$\beta$ minimax is used (see the Greenblatt
reference), many partially analyzed nodes are left behind as the tree is
traversed. For such node, it is known that their value is greater or less than
some number. This information may yield $\alpha$--$\beta$ cutoffs at a node, or at
least remove a node from further consideration.

A game player which uses this procedure along with hash table storage needs to
store the level and value of completely evaluated nodes, or the level and value
range of partially evaluated nodes. Although this could be done with no errors
by storing the board configurations as a key for each node, we might tolerate
some errors if, as a result, operations could be made more efficient. There are
two types of errors worth considering. \underline{Type 1} errors result if keys
are not stored. If two different board configurations yield the same hash
address, the error is not detected and the wrong value may be given to a node
of the lookahead. Type 1 errors also occur if keys are sorted but they are
inadequate to resolve all clashes. If type 1 errors are controlled, the
\underline{type 2} errors may occur. Entries in the hash table may be
overwritten by subsequent entries. Some nodes may have to be evaluated again,
but no wrong values are inserted into the lookahead. We shall give a general
description of the hashin method before considering its application to game
playing.

\section*{Description of the method}

The exclusive or (XOR) operation will be used. Assume the following properities
for the XOR of a random sequence ($r_1$) of n-bit integers:

\begin{itemize}
\item[P1)] $r_i$ XOR $(r_j$ XOR $r_k) = (r_i$ XOR $r_j)$ XOR $r_k$
\item[P2)] $r_i$ XOR $r_j = r_j$ XOR $r_i$
\item[P3)] $r_i$ XOR $r_i = 0$
\item[P4)] if $s_i = r_i$ XOR $r_2$ XOR $\ldots$ XOR $r_i$ then $\{s_i\}$ is a random sequence.
\item[P5)] $\{s_i\}$ is uniformely distributed.
\end{itemize}

Suppose we desire hash codes for the subsets of a finite set $S$. An easy
method would be to assign n-bit integers from a random sequence to the elements
of $S$ and let the hash code of a subset be the XOR of the integers assigned to
the elements of that subset\footnote{Example:
\url{https://gist.github.com/miku/d53e2facdefec281f4de1cb2c219bc6f}}. By P1
and P2 this code is unique, and by P4 and P5 it is randomly and uniformly
distributed. If any element is added to or deleted from a subset, the code
changes by the integer which corresponds to that element.

Other applications result if the objects to be hashed can be üut into a 1-1
relation woth the subsets of a finite set in a natural way. Consider the
problem of hashing English words of ten characters or less. The set S could be
the 260 elements of the form:

\begin{align*}
s_{1,1} &= \text{letter A in position 1} \\
s_{1,2} &= \text{letter A in position 2} \\
&\shortvdotswithin{=}
s_{1,10} &= \text{letter A in position 10} \\
s_{2,1} &= \text{letter B in position 1} \\
&\shortvdotswithin{=}
\end{align*}

Thus every word will correspond to a unique subset of S, since two different
words must differ by some letter at some position. If 260 integers $h_{ij}$
from a random sequence are assigned to the elements of S, then a hash code is
obtained. For example:

\begin{align*}
\text{hash(CAT)} &= h_{3,1} \quad \text{XOR} \quad h_{1,2} \quad \text{XOR} \quad h_{20,3}
\end{align*}

Note that not all subsets of S correspond to legal English words or even to
letter strings, e.g. $\{s_{1,1}, s_{2,1}\}$.

\section*{The clustering of errors}

Although the hash codes are uniformly distributed, the idiosyncrasies of a
particular problem may produce an unusual number of clashes. Continuing our
example for English words, if the following clash occured:

\begin{align*}
\text{hash(CAT)} &= \text{hash(DOG)} \\
\end{align*}

then

\begin{flalign*}
& \text{hash(CAT)} = \text{hash(DOG)} \\
& \text{hash(CATNIP)} = \text{hash(DOGNIP)} \\
& \text{etc.}
\end{flalign*}

Similar problems might occur for any application of XOR hash coding. The basic
method depends upon the uniform distributuion of codes for the subsets of a
finite set S. If two different subsets $S_1$ and $S_2$ have the same hash code,
then for any subset $S_3$ disjoint from $S_1 \cup S_2$ we have

\begin{align*}
\text{hash}(S_1 \cup S_2) &= \text{hash}(S_2 \cup S_3) \\
\end{align*}

and for any subset $S_4 \subset S_1 \cap S_2$ we have

\begin{align*}
\text{hash}(S_1 - S_4) &= \text{hash}(S_2 - S_4) \\
\end{align*}

Hopefully, XOR hash coding will show advantages for some application which
outweights its disadvantages.

\section*{Application to game playing}

Consider the class of games in which men are places or removed from a board,
but are not moved: Go, tic-tac-toe, GO-MOKU, quobic, hex, etc. Suppose that
there are $m$ distinguishable types of men and $n$ locations on the board.

There are at most $m \times n$ possibilities for placing a type of man at a
board loaction, thus every game configuration corresponds to a unique subset of
there $m \times n$ possibilities. If an $m \times n$ array of integers is taken
from a random sequence, then one integer can be assigned to each placement
possibility, and the ash code for a board configuration would be the XOR of all
the men on the board.

If the status of teh game is updated with each move, then only one XOR is
needed for the placement of aman. Since XOR is usually a fast machine
operation, this is probably the fastet hashing method available. By properties
P1-P3 the removal or capture of a man can be accomplished with the same XOR
operation used for its placement. The integer used for placement will cancel
the integer used for removal. During lookahead, the hash codes can be stored
recursively, this only ne XOR operation will be needed for each node visited.

For games of movement such as checkers and chess, a move can be treated as the
removeal of a man at one position and his placement at another. Promotions can
be treated as the removal of one type of man the replacement of another. Chess
and checker moves will typically incove two XOR operations, but such moves as
castling, captures, and multiple captures may require three or more.

The storage requirement for the array of integers is note excessive. Checkers
uses 4 types of men and 32 board locations, chess uses 12 types of men and 64
board locations, and GO uses 2 types of men and 361 board locations. Since only
distunguishable types of men and board locations are noted, this merhod is
guaranteed to recognize all equivalent board configurations as they occur.

Our method has assumed that the moves keading to a board configuration have no
effect upon its evaluation. Bug some states of a game may depend upon the last
move or previous moves. In chess, a pawn is en passant only if it has just
moved, and a king may castle with a root only if both pieces have not
previously moved. In GO, certain captures are forbidden for one turn only. If
such conditions are used in evaluation, then provisions can be made to
incorporate them into the hash code. For example, four integers may be reserved
for the four castling opportunities. When a castling opportunity is remove, its
integer is XOR ed.

If type 1 errors are to be reduced or eliminated, the it is necessary to store
a key which is capable of detecting clashes. However, no description of the
board configuration is produced by our method, asige from the hash code itself.
A simple solution is to produce a second XOR hash code usind an independent
random sequence, and store this second code as a key. If two different board
configurations produce the same primary hash code, the it is unlikely that the
secondary hash codes will be equal as well. The secondary hash code need not
have the same number of bits as the first, which is determinded by the length
of the hash table. If the levels and values stored in the ash table occupy part
of a word, the the remaineder of the word can be used for the secondary hash
code. It is fairly obvious that the greater the number of bits in the secondary
hash code, the less chance there is of type 1 errors.

\section*{Estimation and control of errors}

If two different board configurations produce the same primary and secondary
hash code, the a type 1 error may occur. A wrong value would be inserted into
the move tree search, with some chance of the error backing up to the top of
the tree. Because of the nature of XOR hashing, successors of the two board
configurations will produce an unusual number of type 1 errors as well, making
it difficult to estimate the true probability that na error will affect the
move decision. We shall assume that every type 1 error is a fatal error and
estimate their probability for different key sizes.

Suppose that the hash table has $2^n$ positions and that the keys have $n + m$
bits each. Suppose that during the move tree search, $2^n$ values were
retrieved from the hash table with a probability

\begin{align*}
\frac{2^{n + m} - 1}{2^{n+m}}
\end{align*}

that each was correct. For an apporoximation we assume that these probabilities
are independent (which is conservative in view of the clustering of errors)
yielding the following probability that no errors occur:

\begin{align*}
\text{p(no errors)} = (1 - \frac{1}{2^{n+m}})^{2^n} = (\frac{1}{e})^{2^{-m}} \\
\end{align*}

The approximation can be used to yield the following probabilities:

\begin{center}
\begin{tabular}{c c}
m = number of extra bits & probability of no error \\
0 & 0.36778 \\
1 & 0.60653 \\
2 & 0.77879 \\
3 & 0.88251 \\
4 & 0.93941 \\
5 & 0.96924 \\
6 & 0.98449 \\
7 & 0.99222 \\
8 & 0.99611 \\
9 & 0.99805 \\
10& 0.99901 \\
\end{tabular}
\end{center}

If a game player uses a 32K hash table then 20 bit keys would yield a 3\% error
rate. 25 bit keys would yield .1\% error rate. Many computers have a word size
capable of storing such key as well as the value and level, so a 32K hash table
would fit in 32K words. This control of key size and of the frequency of errors
is a major advantage of the method outlined in this report.

\section*{Comparison with another method}

It is interesting to consider a current hashing method which could be applied
to game playing programs. First obtain an integer which descibes the board
configuration. Then, divide the integer by the hash table size and use the
remainder as a hash address and the quotient as a key. However, the integer
which descibes the board configuration may occupy several (or many) computer
words, thus, the divide will be complecated and slow. The quotient may also
occupy several words, thus most of the hash table would be occupied by these
keys if type 1 errors were to be avoided.

\section*{Conclusion}

A new hashing method has been described, but it is of dubious value for general
applications. Its specific application to game playing (where errors may be
tolerated) shows the following advantages:

\begin{itemize}
\item[1)] It is easy to apply. For example, its implementation for checkers,
chess, and GO are roughly the same.
\item[2)] It is fast.
\item[3)] It enables precise control of the serious errors which result from
the retrieval of wrong information from the hash table.
\item[4)] It allows choice of the size of keys used to control the errors in 3).
\end{itemize}

This method is currently being implemented for two game playing programs
(checkers and chess) at the University of Wisconsin. However, I would be
interested in hearing of any other applications or potential applications it
may have.

\section*{Acknowledgements}

This work was partially supported by NSF grant GJ583 and NIH grant MH12266.

\begin{thebibliography}{9}
\bibitem{knuth}
Knuth, D. \underline{The Art of Computer Programming}, vol. 3 (to appear)
\bibitem{greenblatt}
Greenblatt, R., D. Eastlake, and S. Crocker, The Greenblatt chess program,
\underline{Proceedings Fall Joint Computer Conference}, Thompson, Washington
D.C., vol. 31, Nov. 1967.
\end{thebibliography}
\end{document}

